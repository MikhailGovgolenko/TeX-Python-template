% !TEX program = lualatex
\documentclass[a4paper,12pt]{article}

% ---------------------------
% Разметка страницы
% ---------------------------
\usepackage[a4paper,
    left=2cm, right=2cm,
    top=2cm, bottom=2cm]{geometry}

% ---------------------------
% Языки и шрифты
% ---------------------------
\usepackage{fontspec}
\usepackage{polyglossia}
\setdefaultlanguage{russian}
\setotherlanguage{english}

\setmainfont{CMU Serif}
\setsansfont{CMU Sans Serif}
\setmonofont{CMU Typewriter Text}

% ---------------------------
% Формат даты
% ---------------------------
\renewcommand{\today}{\number\day~\lowercase{\selectlanguage{russian}\monthname[\month]}~\number\year~г.}

% ---------------------------
% Математика
% ---------------------------
\usepackage{unicode-math}
\setmathfont{Latin Modern Math}

% Для \mathcal и \mathbb как в обычном LaTeX
\DeclareMathAlphabet{\mathcal}{OMS}{cmsy}{m}{n}
\let\mathbb\relax
\DeclareMathAlphabet{\mathbb}{U}{msb}{m}{n}

% ---------------------------
% Эпиграф
% ---------------------------
\usepackage{epigraph}
\usepackage{ragged2e}
\renewcommand{\textflush}{flushleft}
\renewcommand{\sourceflush}{flushright}

% ---------------------------
% Ссылки и структура
% ---------------------------
\usepackage[
    colorlinks=true,
    linkcolor=blue,
    urlcolor=blue,
    citecolor=blue
]{hyperref}

\usepackage{perpage} % нумерация сносок по каждой странице
\MakePerPage{footnote}

\usepackage{titlesec} % оформление заголовков
\titleformat{\section}
  {\normalfont\Large\bfseries\filcenter}
  {\thesection}{1em}{}

\usepackage{enumitem} % списки

% ---------------------------
% Базовые пакеты
% ---------------------------
\usepackage{amsmath, indentfirst, float, array, multirow, booktabs, caption, manfnt}

% ---------------------------
% Колонтитулы
% ---------------------------
\usepackage{fancyhdr}
\pagestyle{fancy}
\setlength{\headheight}{15pt}
\fancyhead[L]{}
\fancyhead[C]{\nouppercase{\leftmark}}
\fancyhead[R]{}
\renewcommand{\headrulewidth}{0pt}
\renewcommand{\footrulewidth}{0pt}
\fancyfoot{}   % очищает футер
\renewcommand{\thepage}{}   % отключает номер страницы

% ---------------------------
% TikZ, схемы, графики
% ---------------------------
\usepackage{pgfplots}
\pgfplotsset{compat=1.18}
\usepackage[european,american inductors]{circuitikz}
\usepackage{pgfplotstable}
% при необходимости можно подключить библиотеки TikZ, например:
% \usetikzlibrary{positioning, arrows.meta, calc}

% ---------------------------
% Единицы СИ
% ---------------------------
\usepackage{siunitx}
\InputIfFileExists{siunitx.cfg}{}{}

% ---------------------------
% Изображения и химия
% ---------------------------
\usepackage{subcaption} % современный аналог subfig
\usepackage[version=4]{mhchem}

% ---------------------------
% Цвета, комментарии и типографика
% ---------------------------
\usepackage{comment, bookmark, microtype}
\sloppy
\emergencystretch=1em

% ---------------------------
% Настройка плавающих фигур
% ---------------------------
\def\fps@figure{htbp}

% ---------------------------
% Таблицы
% ---------------------------
\usepackage{csvsimple} % импорт CSV